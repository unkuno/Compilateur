\documentclass[11pt]{article}

\usepackage[french]{babel}
\selectlanguage{french}
\usepackage[latin1]{inputenc}
\usepackage{fancyhdr}
\usepackage{lastpage}
\usepackage{listings}

%%%%%%%%%%
% Taille des pages (A4 serr�)

\setlength{\parindent}{0pt}
\setlength{\parskip}{1ex}
\setlength{\textwidth}{17cm}
\setlength{\textheight}{24cm}
\setlength{\oddsidemargin}{-.7cm}
\setlength{\evensidemargin}{-.7cm}
\setlength{\topmargin}{-.5in}


%%%%%%%%%%
% En-t�tes et pied de pages

\pagestyle{fancyplain}
\renewcommand{\headrulewidth}{0pt}
\addtolength{\headheight}{1.6pt}
\addtolength{\headheight}{2.6pt}
\lfoot{}
\cfoot{}
\rfoot{\footnotesize\sf \thepage/\pageref{LastPage}}
\lhead{\footnotesize\sf Projet GL}
\rhead{\footnotesize\sf Equipe 16} % num�ro d'�quipe Teide 


%%%%%%%%%%
% Informations sur le document

\title{Projet GL\\\emph{Architecture de la partie C}}

\author{BOUSSON Valentin, CONNES C�dric,\\LENTINI S�bastien, NGUY Thomas\\\emph{Equipe 16}}

\date{12 Janvier 2012}


%%%%%%%%%%
% D�but du document

\begin{document}

\maketitle

\section{Pr�sentation de la structure}
La partie C contient :
\begin{itemize}
\item gencode : package g�rant la structure g�n�rale du programme en assembleur
\item gestion\_registres : package permetant la gestion des ressources : allocation, lib�ration...
\item outils\_instruction : package permettant de formatter le texte assembleur : instruction, commentaire...
\item outils\_parcours : package g�rant le parcours dans l'arbre et la traduction des diff�rentes instructions (passe 2 notamment)
\end{itemize} 


gencode et outils\_instruction utilisent les autres packages et donc d�pendent de ces derniers.

\section{D�finition des incr�ments}
La partie C ne contient qu'une passe pour le langage mini d�ca. Ainsi, nous avons d�cid� de faire une d�coupage suivant la grammaire : 
\begin{itemize}
\item print et println
\item litt�raux
\item op�ration arithm�tique et bool�ennes
\item op�ration d'entr�e (lecture)
\item variable
\item structure de controle
\end{itemize}

\section{M�thode de tests}
A l'aide de la test suite, nous avons cr�er des programmes de tests permettant de tester chaque fonction par incr�ments. Nous avons v�rifier l'utilisation et la lib�ration des registres.

\end{document}
