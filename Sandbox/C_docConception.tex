\documentclass[11pt]{article}

\usepackage[french]{babel}
\selectlanguage{french}
\usepackage[utf8]{inputenc}
\usepackage{fancyhdr}
\usepackage{lastpage}
\usepackage{listings}
\usepackage{color}
\usepackage{graphicx}

%%%%%%%%%%
% Taille des pages (A4 serré)

\setlength{\parindent}{0pt}
\setlength{\parskip}{1ex}
\setlength{\textwidth}{17cm}
\setlength{\textheight}{24cm}
\setlength{\oddsidemargin}{-.7cm}
\setlength{\evensidemargin}{-.7cm}
\setlength{\topmargin}{-.5in}


%%%%%%%%%%
% En-têtes et pied de pages

\pagestyle{fancyplain}
\renewcommand{\headrulewidth}{0pt}
\addtolength{\headheight}{1.6pt}
\addtolength{\headheight}{2.6pt}
\lfoot{}
\cfoot{}
\rfoot{\footnotesize\sf \thepage/\pageref{LastPage}}
\lhead{\footnotesize\sf Projet GL}
\rhead{\footnotesize\sf Equipe 16} % numéro d'équipe Teide 


%%%%%%%%%%
% Informations sur le document

\title{Projet GL\\\emph{Conception de la partie C}}

\author{BOUSSON Valentin, CONNES Cédric,\\LENTINI Sébastien, NGUY Thomas\\\emph{Equipe 16}}

\date{10 Janvier 2012}


%%%%%%%%%%
% Début du document

\begin{document}

\maketitle

\section{Choix de conception}
Le package gencode gère la passe 1 et la passe 2. Pour cela, il fait appel aux méthodes 
Creer\_Classe et Ecrire\_Programme.

Creer\_Classe se consacre à la passe 1, à savoir la création du tableau des étiquettes des méthodes et d'insérer la table des méthodes dans la pile.
Pour le moment, seul la classe object est réalisé et par conséquent on ne peut détailler plus précisemment ces étapes.

Ecrire\_Programme quant à lui, à l'aide des fonctions présentes dans le package outils\_parcours, va parcourir l'arbre et en fonction des noeuds va interpréter le programme pour écrire le code assembleur correspondant.

Pour écrire une expression avec la fonction print (ou println), on utilise la fonction Place\_Computation\_Exp afin de pouvoir calculer (ou compiler) l'expression. Cette fonction, récursive, va renvoyer la valeur finale de l'expression, soit sous forme de registre (par ex : "R2"), soit sous forme de valeur immédiate (par ex : "\#2"). Cette valeur renvoyé est une chaine : en effet, bien que nous avons un type pour les registres, nous avons préféré pour des soucis de simplicité utiliser le style Chaine, permettant de traiter indifferement les valeurs immédiates des registres.

La gestion des registres, par le package gestion\_registres, permet l'allocation et la libération (quand demandé) de registres allant de R2 à R15. Comme nous gérons les valeurs dans Place\_Computaton\_Exp avec des Chaines, nous avons réalisé des fonctions dans gestion\_registre permettant de savoir si une chaine représente un registre (Est\_Registre), et de pouvoir libérer un registre via sa représentation sous forme de chaine (ou String) : Liberer\_Via\_String.


\section{Ecriture dans le fichier assembleur}
Toute écriture dans le fichier assembleur doit se faire à l'aide des fonctions 
présentes dans le package outilsv\_instructions : 
\begin{itemize}
\item procedure Ecrit\_Inst(S : String) pour écrire une instruction assembleur
\item procedure Ecrit\_Comment(S : String) pour écrire une commentaire assembleur
\item procedure Ecrit\_Etiq(S : String) pour écrire une étiquette
\end{itemize}

\end{document}
