\documentclass[10pt,a4paper]{article}
\usepackage[T1]{fontenc}
\usepackage[utf8]{inputenc}
\usepackage[french]{babel}
\author{LENTINI Sébastien}
\title{Compte-rendu réunion du 04/01/2012}
\begin{document}
\maketitle

\section{Résumé de la réunion}
Pour notre première réunion, nous avons décidé de définir ensemble une charte de groupe et de travail. Nous avons décidé que LENTINI Sébastien serait le chef de projet de par le fait qu'il connait le mieux chacune des personnes constituant le groupe. Il aurait par conséquent le pouvoir de décision en cas de conflit. Nous avons aussi échangé nos numéros de téléphone pour faciliter la communication (sms possible car chaque membre a sms illimité)

\subsection*{Préférence de travail}
\begin{itemize}
\item Le groupe travaillera ensemble à l'ENSIMAG, avec PC personel (Thomas \& Valentin) ou machine Ensimag (Sébastien \& Cédric)
\item Horraire fixe : 9h-12h / 14h-17h + modulation si nécessaire
\item Réunion le matin à 9h pour définir les tâches de la journée ainsi que leur répartition.
\item Utilisation de la pause du midi pour éventuellement parler de l'avancement et des difficultés que peut rencontrer une personne dans le projet. 
\item En début de projet, nous allons essayer de mixer les personnes du groupe pour renforcer la cohésion du groupe.
\end{itemize}

\subsection*{Convention de codage}

\begin{itemize}
\item Commit fréquent et push quand ça compile, avec un commentaire explicite.
\item Commentaires : \begin{description}
	  \item ads : ce que fait la méthode
	  \item adb : comment on le fait
	  \end{description}
\item Convention de nommage : Ceci\_Est\_Un\_Exemple
\item Aligner in et ou (retour à la ligne) pour faciliter la lecture du code
\item Programmation défensive avec assert et trace à condition (détermination des niveaux à faire)	   
\end{itemize}

\newpage

\subsection*{Planning prévisionnel}
L'objectif est de se familiariser avec le projet et d'avoir dès lundi soir prochain l'étape A de faite.
\begin{description}
\item Jeudi : travail sur la partie A {lexique/syntaxique = 2 groupes}
\item Vendredi : idem
\item Week-end : lecture du dossier + libre service + définition niveau erreur
\end{description}

\end{document}
\end{document}
