\documentclass[11pt]{article}

\usepackage[french]{babel}
\selectlanguage{french}
\usepackage[utf8]{inputenc}
\usepackage{fancyhdr}
\usepackage{lastpage}
\usepackage{tikz}
\usetikzlibrary{patterns}

%%%%%%%%%%
% Taille des pages (A4 serré)

\setlength{\parindent}{0pt}
\setlength{\parskip}{1ex}
\setlength{\textwidth}{17cm}
\setlength{\textheight}{24cm}
\setlength{\oddsidemargin}{-.7cm}
\setlength{\evensidemargin}{-.7cm}
\setlength{\topmargin}{-.5in}


%%%%%%%%%%
% En-têtes et pied de pages

\pagestyle{fancyplain}
\renewcommand{\headrulewidth}{0pt}
\addtolength{\headheight}{1.6pt}
\addtolength{\headheight}{2.6pt}
\lfoot{}
\cfoot{}
\rfoot{\footnotesize\sf \thepage/\pageref{LastPage}}
\lhead{\footnotesize\sf Projet GL}
\rhead{\footnotesize\sf Equipe 16} % numéro d'équipe Teide 

%%%%%%%%%%
% Informations sur le document

\title{Projet Génie Logiciel\\\emph{Documentation de conception}}

\author{BOUSSON Valentin, CONNES Cédric,\\LENTINI Sébastien, NGUY Thomas\\\emph{Equipe 16}}

\date{10 Janvier 2012}


%%%%%%%%%%
% Début du document

\begin{document}

\maketitle

Ce document recense l'ensemble des choix de conception qui ont pu être effectués afin de mettre en place le compilateur Deca. Ces choix peuvent porter sur :
\begin{itemize}
\item l'architecture de l'implémentation (choix de paquetages, ...)
\item l'utilisation des structures disponibles pour telle ou telle fonctionnalité
\end{itemize}

  Nous aborderons les différents points dans l'ordre de la chaîne de compilation (étapes A, B, C).

\section{Etape A : analyse lexicosyntaxique}

Pour cette étape, les outils Aflex et Ayacc ont été utilisés afin de générer de manière systématique un analyseur lexical et un analyseur syntaxique.

\subsection{Analyse lexicale (Aflex)}

L'analyseur lexical est généré par Aflex à l'aide du fichier \verb!lexical.l!. Ce fichier défini les unités lexicales utilisées par le langage Deca (expressions régulières) ainsi que les attributs à synthétiser pour chacune d'entre-elles.

Afin de faciliter l'association d'un lexème à une unité lexicale, un dictionnaire est utilisé (voir \verb!dictionnaire.ads!). Ce choix permet de partager les chaines de caractères conservées en mémoire.

Le dictionnaire peut être initialisé pour prédéfinir les lexèmes associés aux mots réservés et/ou aux caractères spéciaux du langage (car les attributs associés sont de type \verb!Lex_Autre!).

Pour ces deux types d'unité lexicale, il existe donc deux choix :
\begin{enumerate}
\item les insérer au départ dans le dictionnaire et les 'matcher' ensuite par la même règle que les identificateurs
\item les 'matcher' un à un via des règles individuelles
\end{enumerate}

Chaque solution présente à la fois des avantages et des inconvénients. Un choix a donc dû être fait pour l'implémentation :
\begin{itemize}
\item Comme les mots réservés sont reconnus par l'expression régulière utilisée pour les identificateur, ils sont traités avec la première solution.
\item Par contre, afin de ne pas surcharger le dictionnaire, la seconde possibilité est utilisée pour les caractères spéciaux. Cette solution semble satisfaisante, d'autant plus qu'il faudrait modifier l'expression régulière associée aux identificateurs, ce qui serait contre-productif.
\end{itemize}

\subsection{Analyse syntaxique et construction de l'arbre abstrait (Ayacc)}

L'analyseur syntaxique est généré par Ayacc, à l'aide du fichier \verb!syntaxe.y!. Celui-ci vérifie la syntaxe du programme et construit un arbre abstrait qui lui est parfaitement équivalent.

Pour réaliser cette étape, l'arbre est synthétisé, des feuilles (terminaux) vers la racine, en utilisant des attributs (voir \verb!toto!).

Le seul choix d'implémentation porte sur les numéros de lignes. En effet, à chaque noeud de l'arbre est associé à un numéro de ligne, correspondant à sa position dans le programme d'entrée. Or, certains non-terminaux peuvent couvrir plusieurs lignes.

Exemple :

\begin{verbatim}
1 class 
2       A
3         {
4     protected
5               int
6                   x := 6
7                          *
8                            7;
9 }
\end{verbatim}

Le programme ci-dessus est syntaxiquement correct, mais quel est le numéro de ligne de la déclaration de la classe \verb!A! ? de la déclaration de l'attribut \verb!x! ? de l'opération \verb!6 * 7! ?

Pour gérer ce phénomène, des conventions on été adoptées :
\begin{itemize}
\item Pour les opérations binaires, utiliser le numéro de ligne de l'opérateur
\item Pour les autres structures, utiliser le numéro de ligne du mot clé
\end{itemize}

Dans l'expemple ci-dessus, la classe \verb!A! est donc déclarée ligne 1, l'attribut \verb!x! est déclaré ligne 4, et l'opération \verb!6 * 7! est située ligne 7.

\newpage
\section{Etape B : vérifications contextuelles et décorations}

Cette étape permet de vérifier la correction contextuelle du programme d'entrée ainsi que de rajouter à l'arbre abstrait les informations nécessaires (décorations) pour l'étape suivante (génération de code).


  \begin{center}
    \begin{tikzpicture}
      \draw (3,0) rectangle (8, 4);
      \draw (5.5,3.5) node {Outils (+ \verb!GL_Global!)};
      \draw (3,3) -- (8,3);
      \draw (5.5,2.5) node {\verb!Erreurs!};
      \draw (5.5,1.5) node {\verb!Symboles!};
      \draw (5.5,0.5) node {\verb!Verif_Commun!};

      \draw (0,5) rectangle (3,6);
      \draw (1.5,5.5) node {\verb!Verif_Passe1!};
      \draw[->,thick] (5,4) -- (2,5);
      \draw[->,thick] (2,6) -- (5,7);

      \draw (4,5) rectangle (7,6);
      \draw (5.5,5.5) node {\verb!Verif_Passe2!};
      \draw[->,thick] (5.5,4) -- (5.5,5);
      \draw[->,thick] (5.5,6) -- (5.5,7);

      \draw (8,5) rectangle (11,6);
      \draw (9.5,5.5) node {\verb!Verif_Passe3!};
      \draw[->,thick] (6,4) -- (9,5);
      \draw[->,thick] (9,6) -- (6,7);

      \draw (4,7) rectangle (7,8);
      \draw (5.5,7.5) node {\verb!Verif!};
    \end{tikzpicture}
  \end{center}



\subsection{Implémentation en trois passes : \texttt{Verif\_Passe[123]}}

La vérification contextuelle et la décoration de l'arbre abstrait sont réalisées suivant 3 parcours de l'arbre abstrait généré en étape A. \\
Les passes (ou parcours) sont implémentées dans des paquetages distincts, qui définissent un ensemble de procédures de parcours de l'arbre abstrait, ainsi qu'un unique point d'entrée : \verb!Verifier_Decorer_[123]!.
Ces passes remplissent des rôles bien définis suivant la grammaire attribuée du langage Deca.
\begin{itemize}
\item Paquetage \verb!Verif_passe1.adb! : \\Ce paquetage vérifie la validité du nom des classes et la hiérarchie entre elles (extends). On génère un environnement contenant l'environnement prédéfini des types ainsi que ceux des classes définies dans le programme Deca. La décoration des classes sera faite durant la première passe.
\item Paquetage \verb!Verif_passe2.adb! : \\Ce paquetage vérifie la validité des déclarations des champs et la signature des méthodes des classes. On construit un environnement spécifique à chaque classe. La décoration des champs et des méthodes est réalisée durant cette passe, dans laquelle on trouvera notamment son numéro. Enfin on met aussi à jour les informations des classes par son nombre de champs et son nombre de méthodes.
\item Paquetage \verb!Verif_passe3.adb! : \\Ce paquetage vérifie les initialisations et le corps des méthodes. On vérifie aussi le programme principal. Cette passe s'appuie sur les différents environnements crées dans les passes précédentes pour livrer son résultat. On décore aussi le noeud "Retour", les noeuds "Ident" et les noeuds "EXP".
\end{itemize}

L'organisation de chacun de ces trois paquetages est similaire et calquée autant que possible sur la grammaire attribuée correspondante. Dans la majorité des cas,
les procédures sont nommées \verb!Verif_<Nom>!, où ``nom'' est un non-terminal de la grammaire attribuée.


\subsection{Packages "Outils"}

\subsubsection{Opérations communes aux trois passes: \texttt{Verif\_Commun}}
 Le paquetage \verb!Verif_Commun! centralise les diverses opérations qui sont communes aux trois passes, on distingue:
\begin{itemize}
\item Les opérations sur les environnements (union disjointe)
\item Un test de comparaison des signatures des méthodes
\item Les tests sur la compatibilité des types pour la conversion et l'affectation
\item Les tests sur la compatibilité des types pour les opérateurs
\item L'implémentation des règles communes aux 3 grammaires
\end{itemize}

\subsubsection{Environnements prédéfinis : \texttt{Symboles}}

On définit les environnements prédéfinis \verb!Env_Types_Predef! et \verb!Env_Exp_Object! dans le paquetage \verb!Symboles!. \\
L'environnement prédéfini \verb!Env_Types_Predef! contient la définition des types de bases (int, float, boolean) ainsi que celle de void et de la classe Object.
L'environnement prédéfini \verb!Env_Exp_Object! contient l'environnement de la classe Object, à savoir la méthode equals.
Le paquetage fournit également la procédure chargée de rechercher une définition dans un environnement donné. Concrètement, il s'agit simplement de parcourir le chainage des \verb!Table_Defn! jusqu'à trouver l'entrée cherchée (si elle existe).

\subsubsection{Erreurs contextuelles : \texttt{Erreurs}}

La gestion des erreurs pour cette étape a été centralisée dans le paquetage \verb!Erreurs!. Celui-ci définit plusieurs types d'erreur, en fonction du nombre de paramètres nécessaires pour l'affichage (types énumérés \verb!Erreur_[0-3]_Param!).

Les différentes procédures \verb!Afficher_Erreur! permettent alors d'afficher un message d'erreur approprié, paramétré par un numéro de ligne, des paramètres éventuels et un numéro de règle éventuel. 
La liste des erreurs est repertoriée dans le manuel utilisateur.





\newpage
\section{Etape C : génération de code}
L'étape C met en coopération les paquetages suivants :

\begin{center}
  \begin{tikzpicture}[scale=0.5,font=\scriptsize]
    \draw (14,0) rectangle (20,3);
    \draw (17,0) node[above] {\verb!Gestion_Registres!};
    \draw (17,1) node {\verb!Outils_Instructions!};
    \draw (17,2) node[below] {\verb!Compteur_PUSH!};
    \draw (14,2) -- (20,2);
    \draw (17,2.5) node {Outils + \verb!GL_Global!};


    \draw (0,5) rectangle (6,6);
    \draw (3,5.5) node {\verb!Gencode_Classes!};
    \draw[->] (3,6) -- (14,8.5);
    \draw[<-] (3,5) -- (14,2.5);


    \draw (7,5) rectangle (13,6);
    \draw (10,5.5) node {\verb!Gencode_Commun!};
    \draw[<-] (10,5) -- (15,3);
    \draw[->] (12,6) to[bend left, looseness=0.5] (29,6);
    \draw[->] (12,6) to[bend left, looseness=0.5] (22,6);

    \draw[->] (13,5.5) -- (14,5.5);

    \draw (14,5) rectangle (20,6);
    \draw (17,5.5) node {\verb!Gencode_Expressions!};
    \draw[<-] (17,5) -- (17,3);
    \draw[->] (19,6) to[bend left, looseness=0.5] (29,6);

    \draw[->] (20,5.5) -- (21,5.5);

    \draw (21,5) rectangle (27,6);
    \draw (24,5.5) node {\verb!Gencode_Impression!};
    \draw[<-] (24,5) -- (19,3);

    \draw[->] (27,5.5) -- (28,5.5);

    \draw (28,5) rectangle (34,6);
    \draw (31,5.5) node {\verb!Gencode_Programme!};
    \draw[->] (31,6) -- (20,8.5);
    \draw[<-] (31,5) -- (20,2.5);


    \draw (14,8) rectangle (20,9);
    \draw (17,8.5) node {\verb!Gencode!};
  \end{tikzpicture}
\end{center}


\begin{description}
\item[Outils\_Instructions] : paquetage responsable de la construction du programme assembleur, sous la forme d'un chaînage de lignes.

  Ce paquetage permet basiquement d'insérer une nouvelle ligne en fin de code ou d'afficher le programme ainsi construit. Cependant, il s'est vu ajouté les fonctionnalités suivantes :

  \begin{itemize}
  \item centralisation du booléen \verb!Check!, indiquant s'il faut écrire les vérifications de débordement.
  \item inclusion de la séquence \verb!TSTO!, \verb!BOV!, \verb!ADDSP! en début de chaque bloc (ie. programme principal, corps de méthode ou code d'initialisation des champs d'une classe), mise à jour de leurs opérandes lorsque les informations nécessaires deviennent disponibles et ajout éventuel du \verb!SUBSP! en fin de bloc.
  \item insertion de code en début de bloc (utilisé pour la séquence de sauvegarde des registres)
  \end{itemize}

  Ce paquetage permet donc de centraliser et d'abstraire toutes les opérations affectant le chaînage des instructions.

\item[Gestion\_Registres] : paquetage responsable de l'allocation des registres
  
  Son fonctionnement étant complexe, il sera détaillé en section \ref{registres} (page \pageref{registres}).
  
\item[Compteur\_PUSH] : paquetage servant à retenir la place totale utilisée dans la pile pour le bloc courant.
  
  Son fonctionnement est assez simple : chaque incrémentation du sommet de pile (\verb!ADDSP! ou \verb!PUSH!) doit s'accompagne d'un appel à la procédure \verb!Compte_PUSH!. De même, tout appel à \verb!SUBSP! ou \verb!POP! doit s'accompagner d'un appel à \verb!Compte_POP!. Ces deux primitives permettent de maintenir un compteur interne, ainsi que la valeur maximale atteinte par celui-ci depuis le début du bloc.

        Ainsi, en fin d'évaluation du bloc, on peut connaître la place totale nécessaire pour toute l'évaluation de celui-ci (et donc en informer le paquetage \verb!Outils_Instructions! pour mettre à jour les opérandes de \verb!TSTO! et \verb!ADDSP!).
        
      \item[Gencode\_Classes] : paquetage responsable de la passe 1.

        Construit la tables des méthodes et place, dans chaque \verb!Noeud_Classe!, l'opérande contenant l'adresse de la table des méthodes associée.
      \item[Gencode\_Commun] : ensemble des primitives communes à la passe 2.

        Ce paquetage contient notamment :
        \begin{itemize}
        \item Un compteur global pour éviter la répétition des étiquettes
        \item La procédure commune de mise au point (\verb!Mettre_Au_Point!)
        \item Les procédure de création de la méthode \verb!Object.equals! et de la gestion des messages d'erreur
        \item La fonction \verb!Assure_Registre!, qui sera détaillée par la suite
        \end{itemize}

      \item[Gencode\_Expressions] : paquetage responsable de l'évaluation des sous-arbres d'expression.
      \item[Gencode\_Impression] : paquetage responsable de l'impression des données (\verb!print! et \verb!println!).
      \item[Gencode\_Programme] : point d'entrée de la passe 2.

        Ce paquetage gère notamment :
        \begin{itemize}
        \item le parcours des blocs d'instructions (programme principal et corps des méthodes)
        \item les déclarations (de variables et de champs)
        \item l'initialisation des champs
        \item les structures de contrôle (\verb!if! et \verb!while!)
        \end{itemize}

        Il s'agit donc du lien entre tous les paquetages de la passe 2 (\verb!Gencode_Commun!, \verb!Gencode_Expressions! et \verb!Gencode_Impression!).

      \item[Gencode] : point d'entrée de la génération de code.

        Ce paquetage se charge d'initialiser les différents paquetage et de lancer les passes 1 et 2.
\end{description}

\subsection{Passe 1 : construction des tables des méthodes (paquetage \texttt{Gencode\_Classes})}
On distingue 2 sous étapes : 
\begin{itemize}
\item Construction du tableau des étiquettes des méthodes
\item Génération du code pour construire les tables des méthodes
\end{itemize}

Les tables des méthodes sont construites progressivement lors du parcours des classes. Les raisons d'un tel choix seront précisées plus loin.

\subsubsection{Construction du tableau des étiquettes des méthodes}
Afin de pouvoir construire la table des méthodes, il est nécessaire de connaître le nombre de méthodes, leur nom ainsi que l'adresse de la table des méthodes de la super-classe. C'est pourquoi la structure suivante est utilisée :

\begin{center}
  \begin{tikzpicture}
    \draw (0,0) grid[step=4,thick] (14,4);

    \draw (2,4) node[above] {Object};
    \draw (1,1) rectangle (3,3);
    \draw (1,2) -- (3,2);
    \draw (2,2.5) node {@ Object};
    \draw [->,thick] (2,1.5) -- (2,-1);
    
    \draw (0.25,-2) rectangle (3.75,-1);
    \draw (2,-1.5) node {code.Object.equals};

    \draw (6,4) node[above] {A};
    \draw (5,1) rectangle (7,3);
    \draw (5,2) -- (7,2);
    \draw (6,2.5) node {@ A};
    \draw [->,thick] (6,1.5) -- (6,-1);

    \draw (4.25,-4.5) -- (4.25,-1) -- (7.75,-1) -- (7.75,-4.5);
    \draw (6,-1.5) node {code.Object.equals};
    \draw (4.25,-2) -- (7.75,-2);
    \draw (6,-2.5) node {code.A.xxx};
    \draw (4.25,-3) -- (7.75,-3);
    \draw (6,-3.5) node {code.A.yyy};
    \draw (4.25,-4) -- (7.75,-4);

    \draw (10,4) node[above] {B};
    \draw (9,1) rectangle (11,3);
    \draw (9,2) -- (11,2);
    \draw (10,2.5) node {@ B};
    \draw [->,thick] (10,1.5) -- (10,-1);

    \draw (8.25,-3.5) -- (8.25,-1) -- (11.75,-1) -- (11.75,-3.5);
    \draw (10,-1.5) node {code.Object.equals};
    \draw (8.25,-2) -- (11.75,-2);
    \draw (10,-2.5) node {code.B.zzz};
    \draw (8.25,-3) -- (11.75,-3);
    
  \end{tikzpicture}
\end{center}


Cette structure est remplie grâce à une passe dans l'arbre, qui utilise les décors des \verb!Noeud_Ident! afin de récupérer les informations nécessaires (nom de la super classe, nombre de méthodes, numéro de chaque méthode ...). 

Le tableau est géré par le package \verb!Tables!. Cependant, l'ordre de déclaration des classes est important pour construire une table des méthodes cohérente.
Le tableau est donc construit progressivement, afin de ne pas devoir le re-parcourir par la suite.

\subsubsection{Construction de la table des méthodes}

Pour chaque classe, la structure obtenue dans la page précédente ainsi que l'adresse de la super classe sont retournées afin de pouvoir construire la table des méthodes.
Une simple lecture des informations reçues permet, à l'aide des instructions \verb!LEA!, \verb!LOAD! et \verb!STORE!, de générer le code construisant la table des méthodes.

\subsection{Passe 2 : génération du code}
La structure principale de la génération de code est un parcours de l'arbre décoré par l'étape B.
Ce parcours est effectué par les paquetages \verb!Gencode_Expression!, \verb!Gencode_Impression! et \verb!Gencode_Programme!, chacun étant responsable du parcours d'un sous-arbre spécifique.
Ces trois paquetages sont structurés de la même manière, et possèdent chacun :
\begin{itemize}
\item un point d'entrée
\item un ensemble de procédures (ou fonctions) possédant des prototypes similaires
\end{itemize}

\subsubsection{Les expressions (paquetage \texttt{Gencode\_Expressions})}

Pour évaluer les expressions, on utilise des fonctions de la forme :
\begin{verbatim}
  function Place_Xxx (A : in Arbre, ...) return Operande
\end{verbatim}
Ce type de fonction doit se charger d'évaluer le sous-arbre \verb!A! et de retourner l'emplacement du résultat.
Dans le cas où l'emplacement retourné est un registre, une telle fonction se comporte exactement comme la fonction \verb!Allouer! : seul le registre retourné et celui précédemment alloué peuvent être utilisés (soit \verb!RA! et \verb!RB!, voir section \ref{registres}, page \pageref{registres}).

Dans le cas d'un opérateur binaire, le corps de la fonction \verb!Place_Op_Binaire! pourrait être de forme :
\begin{enumerate}
\item Calculer l'opérande gauche (alloue un registre)
\item Cacluler l'opérande droite (alloue un autre registre)
\item Effectuer l'opération et placer le résultat dans le premier registre.

  On peut effectuer ce calcul car l'allocation de l'opérande droite autorise l'utilisation du registre précédemment alloué, c'est à dire l'opérande gauche.
\item Libérer l'opérande droite
\item Retourner l'opérande gauche
\end{enumerate}


\subsubsection{Les impressions (paquetage \texttt{Gencode\_Impression})}

Pour évaluer les impressions (\verb!print! et \verb!println!), on utilise des procédures de la forme :
\begin{verbatim}
  procedure Ecriture_Xxx (A : in Arbre)
\end{verbatim}
Ce type de procédure évalue le sous-Arbre \verb!A!, écrit le code responsable d'imprimer les résultats et libère les ressources utilisées (registres).

\subsubsection{Le corps des blocs (paquetage \texttt{Gencode\_Programme})}

Pour évaluer les autres structures du langage, on utilise des procédures de la forme :
\begin{verbatim}
  procedure Ecrire_Xxx (A : in Arbre)
\end{verbatim}
Comme pour les impressions, ce type de procédure permet d'appeler les paquetages précédents tout en mettant en place la gestion du flux d'exécution.

\subsection{Gestion des registres : paquetage Gestion\_Registres}
\label{registres}

Dans cette partie, on appelle ``registre utilisable'' un registre non scratch, dont la valeur peut être écrasée sans perte d'information. Cela peut être :
\begin{itemize}
\item un registre inutilisé
\item un registre dont le contenu a été préalablement sauvegardé
\end{itemize}

Le paquetage \verb!Gestion_Registres! implémente un gestionnaire de registres, destiné à fournir des registres utilisables au code client, tout en maintenant un état cohérent de la mémoire. Celui-ci est notamment responsable de la sauvegarde des registres dans la pile (ainsi que de leur restauration).

De plus, le gestionnaire est paramétré par le nombre de registres autorisés par l'utilisateur (option -r), ce qui permet de centraliser la gestion de ceux-ci.

\subsubsection{Description du paquetage}
\label{def_reg}

Le fonctionnement externe de ce paquetage est celui d'un allocateur :
\begin{description}
\item[Allouer] demande un registre utilisable au gestionnaire
\item[Liberer] rend le registre spécifié au gestionnaire
\end{description}

A un instant donné, au plus 2 registres peuvent être utilisés par le code client : \verb!RA! (le dernier alloué) et \verb!RB! (l'alloué précédent).

De façon plus formelle, \verb!RA! et \verb!RB! sont définis comme suit :
\begin{itemize}
\item au démarrage, \verb!RA! et \verb!RB! sont indéfinis (ie. non utilisables).
\item après un appel à \verb!Allouer! :
  $\displaystyle\left\{\begin{tabular}{l}
  \verb!RB! = ancien \verb!RA! \\
  \verb!RA! = le registre alloué
\end{tabular}\right.$
\item après un appel à \verb!Liberer! :
  $\displaystyle\left\{\begin{tabular}{l}
  \verb!RA! = ancien \verb!RB! \\
  \verb!RB! = l'alloué précédent \verb!RA!
\end{tabular}\right.$

\end{itemize}


Pour que l'utilisation de ce paquetage soit correcte, il faut (et il suffit) que les appels au deux procédures ci-dessus garantissent les préconditions suivantes :
\begin{enumerate}
\item Ne pas utiliser un autre registre que \verb!RA! ou \verb!RB!. De manière générale, ne pas mentionner explicitement les registres \verb!R2 .. R15! : les utiliser par l'intermédiaire de variables.
\item Toujours restituer les registres dans l'ordre \emph{inverse} de celui-dans lequel ils ont été alloués. Ceci est équivalent à : ne jamais restituer un autre registre que \verb!RA!.
\item Dans le cas d'opérations sur la pile : garantir que tout appel à \verb!Liberer! est effectué dans le même contexte de pile que l'appel à \verb!Allouer! correspondant.
\end{enumerate}

\subsubsection{Implémentation}

Pour conserver son état interne, le paquetage conserve en mémoire :
\begin{description}
\item[Nb\_Alloc] : nombre de registres actuellement alloués
\item[Dernier\_Alloc] : registre alloué le plus récemment, c'est à dire \verb!RA!
\item[Dernier\_Pile] : dernier registre stocké dans la pile
\end{description}

Par convention, les registres sont toujours alloués dans l'ordre croissant (\verb!R2!, \verb!R3!, ...), cycliquement. De même, d'après la règle 2 du paragraphe précédent, ceux-ci doivent être libérés dans l'ordre décroissant.

\emph{Remarque} : d'après la règle mentionée ci-dessus, la connaissance de \verb!RA! rend ``inutile'' le paramètre passé à la procédure \verb!Liberer!. Cependant, cela permet d'effectuer une programmation défensive très efficace : on peut s'assurer que tous les registres sont libérés, et dans le bon ordre.

Pour pouvoir allouer un registre lorsque tous le sont déjà, il suffit de sauvegarder dans la pile la valeur du prochain registre. En effet, c'est normalement celui qui a été alloué le plus tôt, et donc celui qui devrait avoir besoin de sa valeur le plus tard. Par conséquent, sauvegarder ce registre permet de limiter le nombre d'aller-retour dans la pile (\verb!PUSH! / \verb!POP!).

Pour des questions d'optimisation, on peut également essayer de conserver les valeurs sauvegardées dans la pile le plus longtemps possible. En effet, restaurer la valeur d'un registre dès sa libération peut s'avérer inefficace s'il est ré-alloué juste après. Pour pallier à ce problème, le gestionnaire utilise le fait que seules les valeurs de \verb!RA! et \verb!RB! doivent être garanties. En effet, les autres valeurs peuvent ne pas être restaurées, ce qui permet d'économiser un couple \verb!POP! / \verb!PUSH! lors d'une allocation future. On obtient ainsi l'invariant suivant (garanti par \verb!Allouer! et \verb!Liberer!) :

\begin{center}
  \begin{tikzpicture}
    \draw (0,0) grid (8,1);
    \draw (0,0.5) node[left] {Registres};
    \foreach \x in {2,3,...,9} \draw (\x-1.5,1) node[above] {\verb!R\x!};
    \fill[pattern=north east lines] (1,0) rectangle (3,1);
    \draw (1.5,2) node[below] {\verb!RB!};
    \draw (2.5,2) node[below] {\verb!RA!};
    \fill [pattern=dots] (3,0) rectangle (6,1);
    \draw (2.5,-1) node[below] {\verb!Dernier_Alloc!};
    \draw[->] (2.5,-1) -- (2.5,0);
    \draw (5.5,-1) node[below] {\verb!Dernier_Pile!};
    \draw[->] (5.5,-1) -- (5.5,0);

    \draw (9,1.5) rectangle (10,2.5);
    \draw (10,2) node[above right] {valeurs restaurées,};
    \draw (10,2) node[below right] {inutilisables};

    \draw[pattern=north east lines] (9,0) rectangle (10,1);
    \draw (10,0.5) node[right] {valeurs utilisables};

    \draw[pattern=dots] (9,-1.5) rectangle (10,-0.5);
    \draw (10,-1) node[right] {valeurs ``sales''};
  \end{tikzpicture}
\end{center}

Les données marquées ``sales'' correspondent aux registres libérés pour lesquels la valeur intermédiaire n'a pas encore été restaurée. Cette zone est parfaitement connue grace à la variable \verb!Dernier_Pile!.

Intuitivement, on remarque que la zone de données ``sales'' peut contenir au plus \verb!Nb_Reg! - 2 registres (où \verb!Nb_Reg! est le nombre de registres utilisables par le gestionnaire). Cette optimisation est donc d'autant plus efficace que le nombre de registres utilisables est élevé. En particulier, celle-ci n'opère pas s'il n'y a que 2 registres utilisables.

Pour finir, le paquetage fournit une procédure \verb!Purger!, permettant de nettoyer la pile en restaurant l'ensemble des données ``sales''. Ceci s'avère indispensable lors d'un appel de méthode ou pour l'initialisation des champs d'un objet.

\emph{Remarque} : dans certains cas, la valeur de \verb!RB! n'est pas utilisée par le code client, et pourrait donc ne pas être restaurée lors d'un appel à \verb!Liberer! sur le registre \verb!RA!. L'optimisation pourrait donc être poussée, en fournissant une primitive permettant au code client de forcer ou non la restauration de \verb!RB!. Ceci impliquerait de nombreuses modifications du code client, et n'a donc pas encore été implémenté.

\subsubsection{Extensions du paquetage}

Lors d'un appel de méthode ou d'une initialisation de champs, les registres qui seront utilisés doivent être sauvegardés (puis restaurés). Le paquetage a donc été étendu pour permettre de retenir le nombre de registres utilisés (et donc devant être sauvegardés) pendant l'évaluation d'un bloc : \verb!Nb_Util!.
Celui-ci peut être manipulé via les primitives suivantes :
\begin{description}
\item[Reinitialiser\_Registres] : notifie le gestionnaire d'un démarrage de bloc.
\item[Sauvegarder\_Registres] : demande au gestionnaire d'insérer le code permettant la sauvegarde et la restauration des registres. Cette fonction s'appuie notamment sur la procédure \verb!Insere_Sauvegarde_Registre! du paquetage \verb!Outils_Instructions!. Le nombre de registres sauvegardés est retourné afin d'être pris en compte pour la réservation de la place nécessaire au contexte de pile.
\end{description}


\end{document}
