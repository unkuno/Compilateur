Le projet GL fut pour moi l'occasion de prendre des responsabilités. En effet, auparavant, aucun projet à l'Ensimag ne permettait de travailler en groupe de 4 aussi longtemps. Et avec ce projet, j'ai pu, avec l'accord de l'équipe, être chef de projet ce qui m'a donné un aperçu réel du travail que je désire effectuer plus tard.

Durant le projet, je n'ai pas éprouvé de réelle difficulté à planifier les tâches. En effet, j'ai pu recevoir l'aide de toute l'équipe, qui à travers les différentes réunions, m'ont aidé à évaluer les tâches et à les départager suivant les préférences de chacun.
Cependant, le temps que je passais à "planifier" et préparer les suivis était autant de temps à rattraper pour me remettre dans le bain du projet : au début, j'ai eu quelques difficultés à suivre le rythme. Une fois la partie A terminée, ce fut pour moi l'occasion de repartir de bon pied dans le projet : le gros du planning était fait et le modèle des suivis était établi. La partie C, traitant de la génération de code, m'a permis avec l'aide de Valentin de pouvoir enfin trouver ma place dans l'équipe en temps que programmeur. Le reste du projet a été ensuite beaucoup plus agréable. Le sujet ne m'avait dans un premier temps pas vraiment inspiré, mais au fil du temps j'ai de plus en plus apprécié à construire ce compilateur, car il nous a permis de nous améliorer notamment dans le cadre des tests.

Auparavant, j'appréhendais le fait de devenir chef de projet. En effet, on exige souvent de lui à la fois une connaissance technique pointue, mais aussi une capacité d'analyse et d'organisation, chose à laquelle l'Ensimag ne nous a pas encore vraiment préparé. De plus, dans l'équipe, je ne suis pas forcément le plus fort en terme de connaissances techniques, et avoir le rôle de chef m'a paru au début risqué : j'avais peur de perdre ma crédibilité devant les autres.

Cependant, la cohésion de l'équipe m'a permis de surmonté cela et m'a montré qu'au final, on peut très bien s'en sortir sans être le "plus fort". Ma capacité d'organisation, mon aptitude à gérer une équipe ainsi que mon sens du social m'ont aidé à rendre, je l'espère pour tous les membres de mon équipe, le projet GL agréable. 

L'équipe a un très haut niveau de programmation. Je connaissais Valentin et Cédric pour leur sérieux et leur efficacité, et j'ai pu découvrir Thomas qui m'a le plus surpris : alors qu'il vient de Phelma, il a été très efficace et sa persévérance nous a permis de finir ce projet GL à temps.

Pour conclure, je suis fier du travail que nous avons rendu, et fier d'avoir été dans cette équipe. 

