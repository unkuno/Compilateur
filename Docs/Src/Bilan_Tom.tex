Ce projet s'est révélé très enrichissant. Etudiant venant de Phelma, il s'est avéré que nous étions un peu à l'écart durant le 3\up{ème} semestre et ce projet m'a permis de connaître et de travailler avec des personnes ayant intégré l'Ensimag dès la première année. J'apprécie l'initiative de
l'administration qui a imposé une ``mixité'' dans la constitution des groupes et permettant ainsi un ``mélange'', à la fois des personnes et des compétences.

D'un point de vue relationnel, j'en tire un bilan très positif. Mon intégration dans un groupe dont les personnes avaient déjà, dans le passé,
travaillé ensemble s'est très bien déroulée. Bien accueilli, je n'ai pas été mis à l'écart ou ``sous-estimé'' du fait de ne pas avoir fait ma 
première année à l'Ensimag, et j'ai réellement pu prendre part à ce projet. Le travail était conséquent, mais la bonne entente du groupe
et la bonne organisation du travail nous a permis de travailler efficacement, à un rythme convenable. D'un point de vue technique, j'ai 
acquis des connaissances sur de nombreux outils tels que le git ou \LaTeX{} et sur des langages de programmation tel que le bash, afin de générer des scripts. Au final, j'en ressors plus que satisfait de ce projet. 
