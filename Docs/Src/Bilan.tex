\documentclass[11pt]{article}

\usepackage[french]{babel}
\selectlanguage{french}
\usepackage[utf8]{inputenc}
\usepackage{fancyhdr}
\usepackage{lastpage}
\usepackage{listings}
\usepackage{color}
\usepackage{pdfpages}


%%%%%%%%%%
% Taille des pages (A4 serré)

\setlength{\parindent}{0pt}
\setlength{\parskip}{1ex}
\setlength{\textwidth}{17cm}
\setlength{\textheight}{24cm}
\setlength{\oddsidemargin}{-.7cm}
\setlength{\evensidemargin}{-.7cm}
\setlength{\topmargin}{-.5in}


%%%%%%%%%%
% En-têtes et pied de pages

\pagestyle{fancyplain}
\renewcommand{\headrulewidth}{0pt}
\addtolength{\headheight}{1.6pt}
\addtolength{\headheight}{2.6pt}
\lfoot{}
\cfoot{}
\rfoot{\footnotesize\sf \thepage/\pageref{LastPage}}
\lhead{\footnotesize\sf Projet GL}
\rhead{\footnotesize\sf Equipe 16} % numéro d'équipe Teide 

%%%%%%%%%%
% Informations sur le document

\title{Projet Génie Logiciel\\\emph{Bilan}}

\author{BOUSSON Valentin, CONNES Cédric,\\LENTINI Sébastien, NGUY Thomas\\\emph{Equipe 16}}

\date{23 Janvier 2012}


%%%%%%%%%%
% Début du document

\begin{document}

\maketitle

Le bilan est l'occasion pour nous de faire le point sur 1 mois de travail en équipe, de savoir si notre organisation a été efficace ou pas, mais aussi de pouvoir mieux se connaître.

\section{Evolution de l'organisation du projet}
Au début du projet, nous avions défini une charte d'équipe (cf. \verb!Charte.pdf!), formalisant l'organisation ainsi que les règles à respecter. Les réunions en début de journée ont été respectées et très utiles pour la coordination entre équipiers. Les journées de travail en commun ont été suffisantes en début de projet, mais il nous est apparu nécessaire de devoir travailler le week end (notamment le samedi) à l'Ensimag, et les horaires se sont progressivement élargis (9h - 12h45 / 13h30 - 18h, soit environ 2h de plus qu'en début de projet).

L'outil \emph{Planner} nous a été d'une très grande utilité : en effet, après la réalisation du sous-langage ``Hello World'', le fait de décomposer les différentes parties en incréments progressifs nous a permis de mieux jauger la quantité et la durée de travail nécessaire à chacune des parties. Nous avions prévu au tout début du projet de faire évoluer les groupes de travail (Cédric et Thomas sur la partie B, Sébastien et Valentin sur la partie C, puis de changer) mais nous nous sommes aperçus de la difficulté de réaliser cette rotation en cours de projet. Cependant, nous avons progressivement permuté 2 membres, Cédric et Sébastien, afin que ces deux derniers, de par leur connaissance globale du projet, puisse établir des liens entre les deux parties (vérification de l'arbre et génération de code), tandis que 2 spécialistes (Thomas pour la partie B, Valentin pour la partie C) permettait de former rapidement respectivement Sébastien et Cédric sur les parties qu'ils n'avaient pas traité auparavant.

Cette organisation (entre connaissance globale et connaissance pointue) nous a permis d'avancer et de pouvoir apporter un regard neuf sur le code, tout en connaissant les élements dont l'autre partie avait besoin. 


\section{Evolution de l'équipe}

Durant le projet, aucun conflit majeur n'a eu lieu. Le chef de projet, Sébastien, a du trancher à de rares reprises (sur 2 choix proposés par l'équipe), mais globalement l'équipe n'a pas rencontré de difficulté particulière. Afin d'éviter une menace possible que nous avons soulevé en début de projet (division de l'équipe en 2 sous groupe : Valentin et Thomas contre Sébastien et Cédric), nous avons préféré mixer les équipes dès le début. Ainsi, Thomas a travaillé avec Cédric puis avec Sébastien, et Valentin avec Sébastien et Cédric. Cette organisation a permis à chacun d'apprendre à se connaître et à travailler ensemble, et de par le fait d'avoir été ensemble tous les jours pendant près d'1 mois, a créé des liens qui survivront au projet. 


Il est important pour l'équipe de savoir que les objectifs définis en début de projet ont été réalisé, à savoir : 
\begin{quotation}
«Le but de l’équipe est de pouvoir rendre les livrables (maxi déca) en temps et en heure, tout en permettant l’apprentissage des méthodes de gestion d’équipe et en se respectant les uns les autres.»
\end{quotation}

Il nous apparaît aussi intéressant de revenir sur notre analyse de début de projet, en regardant pour chaque élement de la matrice SWOT si il s'est avéré correct ou incorrect.

Les forces et les opportunités de l'équipe se sont avérées exactes. Nous avons formé une bonne équipe, soudée et nous avons réussi à implémenter chacunes des fonctionnalités demandées, de manière propre et efficace.

Concernant les faiblesses, l'absence de vraie compétence oratoire ou écrite s'est fait ressentir. Ainsi, l'écriture des différents documents a pris plus de temps que prévu. Les autres faiblesses de l'équipe, même si elles se sont avérés vraies, n'ont pas été pénalisantes pour le projet : en effet, chacun a fait son possible pour effectuer le plus de tests possible afin de rendre le compilateur le plus débuggé possible, et chacun est content du résultat.

Comme exprimée précédemment, la division du groupe a pu être évitée via un mélange des équipes. 


\section{Le projet GL : un bon moyen d'apprendre sur soi-même et sur les autres}

D'un point de vue technique, il est évident que chacun d'entre nous a appris comment se construit et fonctionne un compilateur.
Cependant, il nous apparaît plus parlant dans ce bilan de s'intéresser à l'aspect SHEME et de ce que nous a appris le projet GL à la fois sur nous et sur les autres. 

Le projet GL demande, de par la quantité de travail, une très bonne organisation. Les suivis et rendus étaient nombreux et nous avons dû préparer ces derniers avec attention. Il nous a fallu à, certains moments, modifier les priorités de certaines tâches afin de pouvoir rendre des documents de suivi à temps : il peut être perturbant de devoir reporter son travail en cours, pour s'attacher à une tâche plus prioritaire. 

Nous avons aussi, dans ce projet, appris à nous faire confiance : en effet, comme la quantité de code à produire est importante, il est clairement impossible qu'une personne vérifie ce que tout le monde fait en plus de s'occuper de sa partie. Une discussion en commun au préalable a permis de vérifier la bonne compréhension du sujet par chacun de nous et a permis ainsi d'éviter de grosses erreurs de conception.
Cette confiance va de paire avec la conception d'équipe : si une faute est commise, il ne sert a rien de gaspiller son énergie à savoir qui est le coupable, car après tout l'erreur est humaine. Il vaut mieux au contraire utiliser notre énergie à la réparation de la faute, ainsi qu'à connaître la raison qui a poussé tel ou tel membre à faire celle-ci, afin de pouvoir éviter que quelqu'un la reproduise plus tard.

La pression imposée par les deadlines du projet peuvent créer des faux-conflits. Une personne, qui croyant bien faire, modifie un fichier à la dernière minute et provoque une erreur est un des risques les plus importants dans ce genre de projet. Nous avons pu éviter ce problème en nous fixant une limite, à savoir de ne plus toucher au dépôt git 1h avant le rendu sans en parler à toute l'équipe. 

Cependant, nous avons subit une autre forme de pression : celle des autres groupes. En effet, au début, nous avons préféré mieux comprendre le sujet que de se lancer tête baisée, et cela nous a fait prendre du retard sur les autres groupes. Cette pression nous a fait douter sur le choix que nous avons fait, et nous a même fait réfléchir à notre efficacité. Cependant nous nous sommes aperçus assez rapidement que cette étape de réfléxion a finalement permis de mieux développer la suite du projet, et surtout d'être plus rapide sur certaines étapes. Nous avons réussi à rattraper assez rapidement le retard et même être en avance sur certains groupes qui étaient au début plus avancés que nous ! Cette avance confortable nous a permis de prendre du temps pour mieux tester notre application, en fin de projet. 

Si le projet GL était à refaire, voici ce que nous aurions conservé :
\begin{itemize}
\item Prendre du temps pour réfléchir au sujet.
\item Favoriser l'esprit de groupe en mélangeant les équipes.
\item User et abuser de \emph{Planner} pour s'organiser.
\item Développer la test-suite dès le début.
\item Travailler à l'Ensimag.
\item Réunion à 9h.
\item Pas de pression entre nous.
\item Utiliser les possibilités de git (branches). 
\item Automatiser les tests via des scripts.
\end{itemize}

et ce que nous aurions changé : 
\begin{itemize}
\item Ecrire des tests avant de commencer à développer pour être plus efficace.
\item Amener nos machines personnelles afin de pouvoir travailler dans des endroits plus calmes (salles du bâtiment D par exemple).
\item Dès le début, mieux découper les tâches sous \emph{Planner} et se fixer des objectifs moins modestes (Maxi-Deca de suite)
\end{itemize}


\section{Bilan personnel}
\subsection{BOUSSON Valentin}
\hspace{1cm}Globalement, j'ai vraiment aimé le sujet traité et les aspects techniques impliqués.
En effet, l'utilisation intensive de l'outil Git et la compréhension du fonctionnement interne d'un compilateur (outil important dans notre domaine) m'ont semblé être des compétences techniques vraiment intéressantes à développer.
Le travail d'équipe a été réussi et très plaisant pour tous, bien que nous avons fait que travailler sur le projet GL pendant près d'1 mois.

\hspace{1cm}Le sérieux de l'équipe m'a beaucoup plu et nous a permis à tous d'avancer selon le planning, sans être soumis à une pression supplémentaire, ce qui n'est pas donné à toutes les équipes (surtout pour un projet scolaire de cette envergure). 
La synergie développée avec Cédric lors des choix de conception et d'optimisation de la partie C a été également vraiment intéressante.

\hspace{1cm}M'ont paru bien moins intéressantes toutes les tâches relatives à la documentation, aux présentations et à l'organisation.
Malgré le fait évident qu'elles sont nécessaires à la communication externe au groupe de travail, leurs applications à ce projet m'ont confortées dans mon choix de carrière : la spécialisation (à savoir préférer la technique et la recherche au social et au management).



\subsection{CONNES Cédric}
Comme évoqué en début de ce document, le projet GL m'a apporté de nouvelles connaissances, à la fois techniques et relationnelles.

\subsubsection{Apports techniques}

D'un point de vue technique, j'ai beaucoup apprécié de travailler sur un compilateur. En effet, les mécanismes de celui-ci m'étaient presque inconnus et m'avaient toujours intrigué. Ayant eu la chance de travailler sur toutes les étapes de compilations, j'ai donc pu combler bon nombre ces manques, et j'en suis très satisfait.

Grâce au travail de Valentin, j'ai également pu me familiariser avec l'élaboration d'une test-suite, ce qui m'a paru très formateur. De plus, j'ai trouvé très intéressant les parties de conception relatives à la génération de code.

J'ai toutefois moins apprécié les aspects répétitifs de certaines parties (notamment les vérifications contextuelles), ainsi que la rédaction des différents documents (bien que nécessaires).

Tout ceci me rassure quant à mon projet professionnel, à savoir me diriger vers de l'expertise technique.

\subsubsection{Apports relationnels}

Pour ma part, la quantité d'apports en terme de gestion d'équipe et de gestion de projet a été bien plus importante que je ne l'avais imaginé.
En effet, les conditions réalistes du projet GL m'ont fait prendre conscience de nombreuses problématiques que je n'avais pas décelées lors de mes précédents projets.

Tout d'abord, j'ai pu toucher du doigt les contraintes liées aux ``deadlines'' et donc à la gestion de planning. De plus, l'ampleur du sujet m'a appris à découper les tâches ainsi qu'à évaluer la quantité de travail nécessaire pour chaque incrément. Bien que cela reste pour moi une tâche difficile, je pense avoir beaucoup appris dans ce domaine.

Concernant le travail de groupe et les relations entre équipiers, je suis également très content : l'équipe a fait preuve de cohésion et de sérieux, et une bonne émulation est restée présente tout au long du projet.

Je suis donc globalement satisfait du déroulement du projet. Je pense que les connaissances emmagasinées durant ces trois semaines me seront largement utiles quant à l'appréhension de mes futurs projets.

\subsection{NGUY Thomas}
Ce projet s'est révélé très enrichissant. Etudiant venant de Phelma, il s'est avéré que nous étions un peu à l'écart durant le 3\up{ème} semestre et ce projet m'a permis de connaître et de travailler avec des personnes ayant intégré l'Ensimag dès la première année. J'apprécie l'initiative de
l'administration qui a imposé une ``mixité'' dans la constitution des groupes et permettant ainsi un ``mélange'', à la fois des personnes et des compétences.

D'un point de vue relationnel, j'en tire un bilan très positif. Mon intégration dans un groupe dont les personnes avaient déjà, dans le passé,
travaillé ensemble s'est très bien déroulée. Bien accueilli, je n'ai pas été mis à l'écart ou ``sous-estimé'' du fait de ne pas avoir fait ma 
première année à l'Ensimag, et j'ai réellement pu prendre part à ce projet. Le travail était conséquent, mais la bonne entente du groupe
et la bonne organisation du travail nous a permis de travailler efficacement, à un rythme convenable. D'un point de vue technique, j'ai 
acquis des connaissances sur de nombreux outils tels que le git ou \LaTeX{} et sur des langages de programmation tel que le bash, afin de générer des scripts. Au final, j'en ressors plus que satisfait de ce projet. 

\subsection{LENTINI Sébastien}
Le projet GL fut pour moi l'occasion de prendre des responsabilités. En effet, auparavant, aucun projet à l'Ensimag ne permettait de travailler en groupe de 4 aussi longtemps. Et avec ce projet, j'ai pu, avec l'accord de l'équipe, être chef de projet ce qui m'a donné un aperçu réel du travail que je désire effectuer plus tard.

Durant le projet, je n'ai pas éprouvé de réelle difficulté à planifier les tâches. En effet, j'ai pu recevoir l'aide de toute l'équipe, qui à travers les différentes réunions, m'ont aidé à évaluer les tâches et à les départager suivant les préférences de chacun.
Cependant, le temps que je passais à "planifier" et préparer les suivis était autant de temps à rattraper pour me remettre dans le bain du projet : au début, j'ai eu quelques difficultés à suivre le rythme. Une fois la partie A terminée, ce fut pour moi l'occasion de repartir de bon pied dans le projet : le gros du planning était fait et le modèle des suivis était établi. La partie C, traitant de la génération de code, m'a permis avec l'aide de Valentin de pouvoir enfin trouver ma place dans l'équipe en temps que programmeur. Le reste du projet a été ensuite beaucoup plus agréable. Le sujet ne m'avait dans un premier temps pas vraiment inspiré, mais au fil du temps j'ai de plus en plus apprécié à construire ce compilateur, car il nous a permis de nous améliorer notamment dans le cadre des tests.

Auparavant, j'appréhendais le fait de devenir chef de projet. En effet, on exige souvent de lui à la fois une connaissance technique pointue, mais aussi une capacité d'analyse et d'organisation, chose à laquelle l'Ensimag ne nous a pas encore vraiment préparé. De plus, dans l'équipe, je ne suis pas forcément le plus fort en terme de connaissances techniques, et avoir le rôle de chef m'a paru au début risqué : j'avais peur de perdre ma crédibilité devant les autres.

Cependant, la cohésion de l'équipe m'a permis de surmonté cela et m'a montré qu'au final, on peut très bien s'en sortir sans être le "plus fort". Ma capacité d'organisation, mon aptitude à gérer une équipe ainsi que mon sens du social m'ont aidé à rendre, je l'espère pour tous les membres de mon équipe, le projet GL agréable. 

L'équipe a un très haut niveau de programmation. Je connaissais Valentin et Cédric pour leur sérieux et leur efficacité, et j'ai pu découvrir Thomas qui m'a le plus surpris : alors qu'il vient de Phelma, il a été très efficace et sa persévérance nous a permis de finir ce projet GL à temps.

Pour conclure, je suis fier du travail que nous avons rendu, et fier d'avoir été dans cette équipe. 



\end{document}

